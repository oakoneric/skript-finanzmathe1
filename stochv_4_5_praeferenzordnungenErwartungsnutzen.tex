\section{Präferenzordnungen und Erwartungsnutzen}

\textbf{Kritik an Markowitz:}
\begin{itemize}[nolistsep, topsep=-\parskip]
	\item Standardabweichung $\sqrt{\Var[R]}$ nicht unbedingt gutes Risikomaß
	\item Entwicklungen unter Unsicherheit meist komplexer als durch Erwartungswert-Varianz-Prinzip beschrieben.
\end{itemize}

\vspace{\parskip}

\fbox{\textbf{Axiomatischer Zugang:}} Präferenzordnungen (PO) ---
Sei $(\Omega, \F, \P)$ ein Wahrscheinlichkeitsraum und wie üblich $L_1(\Omega, \F, \P)$ der Raum der integrierbaren Zufallsvariablen. Setze
\begin{equation*}
	\mathcal{M} \defeq \text{Menge der Verteilungsfunktionen } F_X \text{ von } X \in L_1(\Omega, \F, \P)
\end{equation*}
Seien $X,Y \in L_1(\Omega)$ mit der Interpretation als risikobehaftete Auszahlungen (\enquote{Lotterie}) versehen. Wir wollen eine Ordnungsrelation $\unlhd$ mit Bedeutung
\begin{equation*}
	X \unlhd Y \equivalent \text{\enquote{$Y$ wird bevorzugt gegenüber $X$}}
\end{equation*}
Wir beschränken uns auf \enquote{verteilungsinvariante} Präferenzordnungen (unabhängig von der Verteilung), welche durch Relation auf $\mathcal{M}$ erklärt werden, d.h.
\begin{equation*}
	X \unlhd Y \equivalent F_X \unlhd F_Y
\end{equation*}

\begin{*definition}
	Eine Relation $\unlhd$ auf $\mathcal{M}$ heißt \begriff{Präferenzordnung}, wenn gilt
	\begin{align*}
		F \unlhd F \quad &\forall F \in \mathcal{M} \tag{Reflexivität} \\
		(F \unlhd G) \land (G \unlhd H) \follows (F \unlhd H) \quad &\forall F, G, H \in \mathcal{M} \tag{Transitivität} \\
		(F \unlhd G) \lor (G \unlhd F) \quad &\forall F,G \in \mathcal{M} \tag{Vollständigkeit}
	\end{align*}
\end{*definition}
\begin{*bemerkung}
	\begin{itemize}[nolistsep]
		\item Menge $\mathcal{M}$ ist konvex, d.h. für alle $F,G \in \mathcal{M}$ und $\alpha \in [0,1]$ gilt
		\begin{equation*}
			H \defeq (1-\alpha) F + \alpha G \in \mathcal{M}
			\tag{$\star$} \label{eq: praeferenz-star}
		\end{equation*}
		\item \eqref{eq: praeferenz-star} lässt sich als ''Mischen`` von $F$ und $G$ interpretieren.
		\item Sei $X \sim F_X$ und $Y \sim F_Y$ sowie $A \upmodels (X,Y)$ mit $\P(A = 0) = \alpha$ und $\P(A = 1) = (1-\alpha)$. Dann gilt
		\begin{equation*}
			(1-A) X + A Y \sim (1-\alpha) F_X + \alpha F_Y
		\end{equation*}
		\item Aus gegebener Präferenzordnung können wir ableiten:
		\begin{itemize}
			\item Äquivalenzrelation $F \sim G \equivalent (F \unlhd G) \land (G \unlhd F)$ -- ''Indifferenz zwischen $F$ und $G$``
			\item strikte Relation: $F \lhd G \equivalent (F \unlhd G) \land \lnot (G \unlhd F)$ -- ''$G$ wird strikt gegenüber $F$ bevorzugt``
		\end{itemize}
		\item Für ''deterministische Zufallsvariable`` $a \in \R$ ist die Verteilungsfunktion $F_a = \one_{[a,\infty)}$.
	\end{itemize}
\end{*bemerkung}

Eine Präferenzordnung kann folgende Eigenschaften besitzen:
\begin{enumerate}[label=(\Roman*)]
	\item \textbf{Monotonie:} Für alle $a,b \in \R$ mit $a \le b$ gilt $F_a \unlhd F_b$ (''mehr besser als weniger``)
	\item \textbf{Risikoaversion:} Für alle $X \in L_1(\Omega)$ gilt $F_X \unlhd F_{\EW[X]}$ (''sicher besser als unsicher``)
	\item \textbf{Mittelwertseigenschaft:} Seien $F,G,H \in \mathcal{M}$ mit $F \unlhd G \unlhd H$. Dann existiert ein $\alpha \in [0,1]$ mit $(1-\alpha) F + \alpha H \sim G$.
	\item \textbf{Unabhängigkeitsaxiom:} Für alle $F, G, H \in \mathcal{M}$ und alle $\alpha \in [0,1]$ gilt
	\begin{equation*}
		F \unlhd G \follows (1-\alpha) F + \alpha H \unlhd (1-\alpha) G + \alpha H
	\end{equation*}
\end{enumerate}

\begin{*definition}[Erwartungsnutzen]
	Sei $\abb{U}{\R}{[-\infty, \infty]}$ monoton steigend und konkav (''(Bernoulli'sche) Nutzenfunktion``).  Dann definiert
	\begin{equation*}
		F \unlhd_U G \equivalent \int U \diff{F} \le \int U \diff{G}
		\tag{$\star$} \label{eq: erwartungsnutzen-star}
	\end{equation*}
	eine Präferenzordnung auf $\mathcal{M}$. Wir sagen ''$\unlhd_U$`` folgt dem Erwartungsnutzenprinzip (ENP).
\end{*definition}

\begin{*bemerkung}
	Wenn wir \eqref{eq: erwartungsnutzen-star} für Zufallsvariablen $X,Y \in L_1$ formulieren, erhalten wir 
	\begin{equation*}
		X \unlhd_U Y \equivalent \EW[U(X)] \le \EW[U(Y)]
	\end{equation*}
\end{*bemerkung}

\begin{*beispiel}
	Wichtige Beispiele für Nutzenfunktionen:
	\begin{itemize}
		\item \textbf{logarithmischer Nutzen:}
		$U(x) = \begin{cases}
		\log(x) & \text{für } x > 0 \\ - \infty &\text{für } x \le 0
		\end{cases}$
		\item \textbf{Potenznutzen:}
		$U(x) = \begin{cases} \frac{x^{1-\alpha}}{1 - \alpha} &\text{für } x \ge 0 \\ - \infty &\text{für } x \le 0
		\end{cases}$ mit $\alpha \in (0,\infty) \setminus \menge{1}$
		\item \textbf{Exponentialnutzen:} $U(x) - \gamma \exp(-\gamma x)$ mit $\gamma \in \R$.
	\end{itemize}
\end{*beispiel}

\begin{theorem}[Satz von \person{von Neumann} \& \person{Morgenstern}, 1953]
	\label{theorem: 4.6}
	Sei $\card{\Omega} < \infty$ und $\unlhd$ eine Präferenzordnung auf $\mathcal{M}_\Omega$. Dann sind äquivalent:
	\begin{enumerate}
		\item Die Präferenzordnung erfüllt (I) bis (IV)
		\item Die Präferenzordnung folgt dem Erwartungsnutzenprinzip.
	\end{enumerate}
\end{theorem}
\begin{proof}
	Wir zeigen hier nur die Rückrichtung, d.h. $\unlhd$ erfüllt die Eigenschaften (I) bis (IV).
	\begin{enumerate}[label=(\Roman*)]
		\item Sei $a \le b$. Dann gilt $\int U \diff{F_a} = U(a) \overset{U \text{ steigend}}{\le} U(b) = \int U \diff{F_b}$, also $F_a \unlhd F_b$.
		\item Sei $X \in L_1$. Es gilt
		\begin{equation*}
			\int U \diffskip{F_X} 
			= \int_{-\infty}^\infty U(x) \diffskip{F_X} \overset{\text{Jensen}}{\le} U \brackets{\int_{-\infty}^\infty x \diffskip{F_X} } 
			= U \brackets{\EW[X]} = \int U \diff{F_{\EW[X]}}
		\end{equation*} 
		und somit $F_X \unlhd F_{\EW[X]}$ (oder alternativ $\EW[U(X)] \le U(\EW[X])$ mit der Jensen-Ungleichung).
		\item Sei $F \unlhd G \unlhd H$. Zu zeigen: es existiert ein $\alpha_\ast \in [0,1]$ mit $(1-\alpha_\ast) F + \alpha_\ast H \sim G$.
		Setze
		\begin{equation*}
			\alpha_\ast \defeq \frac{\int U \diffskip{G} - \int U \diffskip{F}}{\int U \diffskip{H} - \int U \diffskip{F}}
		\end{equation*}
		Es gilt 
		\begin{equation*}
			\begin{aligned}
				\int U ((1-\alpha_\ast) \diff{F} + \alpha_\ast \diff{H}) 
				&= (1-\alpha_\ast) \int U \diff{F} + \alpha_\ast \int U \diff{H} \\
				&= \int U \diff{F} + \alpha_\ast \brackets{\int U \diff{H} - \int U \diff{F}} \\
				&= \int U \diff{G}	
			\end{aligned}
		\end{equation*}	
		\item zu zeigen: $F \unlhd G \follows (1-\alpha) F + \alpha H \unlhd (1-\alpha) G + \alpha H$
		\begin{equation*}
			\begin{aligned}
			\int U ((1-\alpha) \diff{F} + \alpha \diff{H}) 
			&= (1-\alpha) \int U \diff{F} + \alpha \int U \diff{H} \\
			\overset{F \unlhd G}&{\le} (1-\alpha)  \int U \diff{G} + \alpha \int  U \diff{H} \\
			&= \int U ((1-\alpha) \diff{G} + \alpha \diff{H})
			\end{aligned}
		\end{equation*}
	\end{enumerate}
	
\end{proof}

\begin{*bemerkung}
	\begin{itemize}[nolistsep]
		\item \cref{theorem: 4.6} wurde in den 50ern als starke Rechtfertigung für das Erwartungsnutzenprinzip wahrgenommen.
		\item (Empirische) Kritik Ende der 70er durch die Psychologen \person{Kahnemann} und \person{Tversky} \\
		$\Rightarrow$ ''Prospect Theory``
	\end{itemize}
\end{*bemerkung}

\begin{*definition}
	Sei $U$ eine Bernoullische Nutzenfunktion und $\unlhd_U$ die zugehörige Präferenzordnung.
	\begin{enumerate}[label=(\alph*)]
		\item Für $X \in L_1$ heißt $c \defeq c_\ast(X,U) \in \R$ mit $c \sim_U X$ das \begriff{certainty equivalent} von $X$.
		\item Für $U \in C^2$ heißen
		\begin{equation*}
			A_U(x) \defeq - \frac{U''(x)}{U'(x)} \qquad R_U(x) \defeq -\frac{x U''(x)}{U'(x)}
		\end{equation*}
		die \begriff{\person{Arrow}-\person{Pratt}-Koeffizienten} der absoluten bzw. relativen Risikoaversion.
	\end{enumerate}
\end{*definition}

\begin{*bemerkung}
	\begin{itemize}
		\item Für $U$ streng monoton steigen gilt $c_\ast(X,U) = U^{-1}(\EW[U(X)])$, denn
		\begin{equation*}
			c \sim_U X \equivalent U(c) = \EW[U(X)]
		\end{equation*}
		\item Motivation für $A_U(x)$: Einer Person mit Vermögen $x \in \R$ werde eine ''Lotterie`` $\epsilon Y$ angeboten (mit $\epsilon$ klein). Nach dem Erwartungsnutzenprinzip sollte die Person diese annehmen, wenn $\EW[U(x + \epsilon Y)] \ge U(x)$. Taylor-Entwicklung liefert
		\begin{equation*}
			\EW[U(x + \epsilon Y)] - U(x) = \EW[\epsilon Y] U'(x) + \frac{1}{2} \EW[\epsilon^2 Y^2] U''(x) + \dots
		\end{equation*}
		Die Person sollte also annehmen, wenn 
		\begin{equation*}
			2 * \frac{\EW[\epsilon Y]}{\EW[\epsilon^2 Y^2]} \ge - \frac{U''(x)}{U'(x)} = A_U(x) \qquad\qquad \textcolor{cdgray}{\frac{\text{erw. Gewinn}}{\text{Risiko}}}
		\end{equation*}
	\end{itemize}
\end{*bemerkung}


\fbox{\parbox{\linewidth}{\textbf{\fosfamily Klassisches Beispiel zum ''certainty equivalent`} \\ \textbf{St. Petersburger Paradoxon} (\person{Nicholas Bernoulli}, 1713)}}



\begin{description}
	\item[Spiel:] Es werde eine faire Münze solange geworfen bis in der $N$-ten Runde das erste Mal ''Zahl`` fällt. Der Gewinn beträgt $2^{N-1}$ Euro.
	%
	\item[Frage:] Wie hoch soll der Einsatz sein um am Spiel teilzunehmen?
	%
	\item[Antwort 1:] Einsatz = erwarteter Gewinn: $N$ ist geometrisch verteilt, d.h. $\P(N = k) = \brackets{\frac{1}{2}}^k$ für $k \in \N$.
	\begin{equation*}
		\EW[2^{N-1}] 
		= \sum_{k=1}^\infty \brackets{\frac{1}{2}}^k 2^{k-1} 
		= \sum_{k=1}^\infty \brackets{\frac{1}{2}} 
		= + \infty
	\end{equation*}
\end{description}

Die Antworten 2 und 3 von Gabriel Cramer und Daniel Bernoulli entsprechen dem certainty equivalent mit Nutzenfunktionen
\begin{align*}
		U_1(x) &= \sqrt{x} \tag{Cramer} \\
		U_2(x) &= \log(x) \tag{Bernoulli}
\end{align*}

\begin{description}
	\item[Antwort 2 von Cramer:] Es ist $c_1 = c_\ast (2^{N-1}, U_1) = U_1^{-1} \brackets{\EW[U_1(2^{N-1})]}$ und
	\begin{equation*}
		\begin{aligned}
			\EW[U_1(2^{N-1})] 
			&= \sum_{k=1}^\infty \brackets{\frac{1}{2}}^k \sqrt{2^{k-1}} 
			= \sum_{k=1}^\infty 2^{-k + \frac{k-1}{2}} 
			= \frac{1}{\sqrt{2}} \sum_{k=1}^\infty 2^{-\frac{k}{2}} \\
			&= \frac{1}{\sqrt{2}} \sum_{k = 1}^\infty \brackets{\frac{1}{\sqrt{2}}}^k \\
			&= \frac{1}{\sqrt{2}} * \frac{\frac{1}{\sqrt{2}}}{1 - \frac{1}{\sqrt{2}}}
			= \frac{1}{2 - \sqrt{2}}
		\end{aligned}
	\end{equation*}
	Somit also $c_1 = \brackets{\frac{1}{2-\sqrt{2}}}^2 \approx 2.914$, d.h. der faire Einsatz beträge $2.91$ Euro.
	%
	\item[Antwort 3 von Bernoulli:] $c_2 = c_\ast (2^{N-1}, U_2)$.
	\begin{equation*}
		\begin{aligned}
			\EW[U_2(2^{N-1})] 
			= \sum_{k=1}^\infty \brackets{\frac{1}{2}}^k \log\brackets{2^{k-1}} 
			&= \log(2) * \sum_{k=1}^\infty 2^{-k} (k-1) \\
			&= \frac{\log(2)}{2} * \sum_{k=0}^\infty k * 2^{-k} \\
			\overset{(\star)}&{=} \frac{\log(2)}{4} * \frac{1}{(1- \frac{1}{2})^2} \\
			&= \log(2)
		\end{aligned}
	\end{equation*}
	wobei wir in $(\star)$ verwendet haben, dass
	\begin{equation*}
		\sum_{k=0}^\infty z^k = \frac{1}{1-z} \overset{\partdiff{z}}{\follows} \sum_{k=0}^\infty k * z^{k-1} = \frac{1}{(1-z)^2}
	\end{equation*}
	Somit gilt $c_2 = e^{\log(2)} = 2$, d.h. der faire Einsatz beträgt $2$ Euro.
\end{description}
