\section{Die Markowitz-Modelle}

\subsection{Markowitz-Modell 1: Portfoliooptimierung ohne risikofreie Anlage}

Wir betrachten Anlagegüter $S = (S^1, \dots, S^n)$ mit stochastischen Einperioden-Renditen $R=\trans{(R^1, \dots, R^n)}$, d.h. $S_T^i = S_0^i (1+R^i)$ ($i \in [n]$), mit
\begin{itemize}
	\item Erwartungswert $\mu = \EW[R] = \trans{(\mu_1, \dots, \mu_n)} \in \Rn$
	\item Kovarianzmatrix $\Sigma = \EW[(R - \mu) * \transpose{R - \mu}]$\footnote{$\Sigma_{ii} = \Var[R^i]$ und $\Sigma_{ij} = \Cov{R^i}{R^j}$} \\
	Annahme: $\Sigma$ ist regulär (und ist positiv definit), d.h. $\Sigma^{-1}$ existiert
\end{itemize}

Das \textbf{Ziel} ist nun, das Anlagevermögen\footnote{Wir skalieren das Vermögen stets aus $1$, daraus ergeben sich keine Einschränkungen.} $w=1$ auf Anlagegüter $S^1, \dots, S^n$ aufzuteilen. 
Mit $p_i$ bezeichnen wir die \textbf{Investition} in $S^i$, d.h. es muss stets $p_1 + \dots + p_n = w = 1$ gelten.

Erwartete Rendite:
\begin{equation*}
	\mu_p = \EW[\trans{p} R] = \trans{p} \mu
\end{equation*}

Risiko (Standardabweichung): 
\begin{equation*}
	\sigma_p = \sqrt{\Var[\trans{p}R]} = \sqrt{\EW[(\trans{p} (R-\mu))^2]} = \sqrt{\EW[\trans{p}(R-\mu)\transpose{R-\mu} p]} = \sqrt{\trans{p} \Sigma p}
\end{equation*}

\fbox{\textbf{Optimales Anlageproblem:}}
minimiere das Risiko gegeben einer Zielrendite $\mu_\ast$
\begin{equation*}
	\begin{aligned}
	&\min \tfrac{1}{2} \trans{p} \Sigma p \text{ über } \Rn  \\
	\text{unter NB } &\trans{p} \mu = \mu_\ast \und \trans{p} \one = 1 \mit \one = \transpose{1, \dots, 1} \in \Rn
	\end{aligned}
	\tag{Mark-1} \label{eq: mark-1}
\end{equation*}

Die Lagrange-Zielfunktion ergibt sich zu 
\begin{equation*}
	\mathcal{L}(p,\lambda_1,\lambda_2) = \frac{1}{2} \trans{p} \Sigma p + \lambda_1 (\mu_\ast - \trans{p} \mu) + \lambda_2 (1-\trans{p} \one)
\end{equation*}
mit dualer Zielfunktion
\begin{equation*}
	g(\lambda_1, \lambda_2) = \inf_{p \in \Rn} \mathcal{L}(p, \lambda_1, \lambda_2)
\end{equation*}

Für den Gradienten der Zielfunktion gilt
\begin{equation*}
	\begin{aligned}
		\nabla_p \mathcal{L}(p, \lambda_1, \lambda_2) &= \Sigma p - \lambda_1 \mu - \lambda_2 \one \overset{!}{=} 0 \\
		\follows p_\ast &= \Sigma^{-1} (\lambda_1 \mu + \lambda_2 \one)
	\end{aligned}
\end{equation*}
d.h 
\begin{align*}
	g(\lambda_1, \lambda_2) 
	&= \mathcal{L}(p_\ast, \lambda_1, \lambda_2) \\
	&= \frac{1}{2} \trans{(\lambda_1 \mu + \lambda_2 \one)} \Sigma^{-1} \Sigma \Sigma^{-1} (\lambda_1 \mu + \lambda_2 \one) - \transpose{\lambda_1 \mu + \lambda_2 \one} \Sigma^{-1} (\lambda_1 \mu + \lambda_2 \one) + \lambda_1 \mu_\ast + \lambda_2 \\
	&= -\frac{1}{2} \brackets{\lambda_1^2 a + 2 \lambda_1 \lambda_2 b + \lambda_2^2 c} + \lambda_1 \mu_\ast + \lambda_2
\end{align*}
mit
\begin{equation*}
	a = \trans{\mu} \Sigma^{-1} \mu \quad b = \trans{\mu} \Sigma \one \qquad c = \trans{\one} \Sigma \one
\end{equation*}
Es gilt $a \ge 0$, $c \ge 0$ und mit Cauchy-Schwarz auch $ac \ge b^2$.

Maximiere $g$.
\begin{align*}
	\partdiff{\lambda_1} g &= -a \lambda_1 - b \lambda_2 + \mu_\ast \overset{!}{=} 0 &\equivalent&& a \lambda_1 + b \lambda_2 &= \mu_\ast \\
	\partdiff{\lambda_2} g &= -b \lambda_1 -c \lambda_2 + 1 \overset{!}{=} 0 &\equivalent&& b \lambda_1 + c \lambda_2 &= 1
\end{align*}
Wir nehmen an, dass $ac > b^2$.
\begin{itemize}
	\item $-b$I $+a$II: $(ac - b^2) \lambda_2 = a - b \mu_\ast \follows \lambda_2^\ast = \frac{a - b \mu_\ast}{ac - b^2}$. 
	\item $c$I$-b$II: $(ac - b^2) \lambda_1 = c\mu_\ast - b \follows \lambda_1^\ast = \frac{c \mu_\ast - b}{ac - b^2}$
\end{itemize}

Minimierer von \eqref{eq: mark-1}:
\begin{equation*}
	p_\ast = \lambda_1^\ast \Sigma^{-1} \mu + \lambda_2^\ast \Sigma^{-1} \one
\end{equation*}

\begin{korollar}[Tobin's Two-Fund-Separation]
	Jedes Pareto-optimale Portfolio für \eqref{eq: mark-1} kann (unabhängig von $n$) als Linearkombination der zwei Portfolios
	\begin{equation*}
		p_1^\ast = \Sigma^{-1} \mu \quad \und \quad p_2^\ast = \Sigma^{-1} \one
	\end{equation*}
	dargestellt werden.
\end{korollar}
Dabei kann man $p_1^\ast$ als das renditeorientierte und $p_2^\ast$ als das sicherhheitsorientierte (das Risiko minimierende) Portfolio beschreiben.

\begin{*bemerkung}
	\begin{itemize}[nolistsep]
		\item Die Gewichtung der zwei Portfolios $p_1^\ast$ und $p_2^\ast$ orientiert sich am Renditeziel $\mu_\ast$. 
		\item Die Portfolios $p_1^\ast$ und $p_2^\ast$ sind breit diversifiziert\footnote{Die Vektoren $p_1^\ast$ und $p_2^\ast$ sind also nicht dünnbesetzt und haben in der Regel keine Null-Einträge.}, d.h. sie nutzen alle Anlagegüter $S = (S^1, \dots, S^n)$. 
		\item $p_1^\ast$ und $p_2^\ast$ kann man auch als Anlagefonds interpretieren, welche Vermögen entsprechend den Portfolios $p_1^\ast$, $p_2^\ast$ anlegen. Diese zwei Fonds sind ausreichend um (unabhängig von $\mu_\ast$) Vermögen Pareto-optimal zu investieren.
	\end{itemize}
\end{*bemerkung}

Zuletzt wollen wir noch das Risiko der optimalen Strategie $p_\ast$ berechnen.
\begin{equation*}
	\begin{aligned}
		\sigma_\ast^2 &= \Var[\trans{p_\ast} R] = \EW[(\trans{p_\ast} (R - \mu))^2] = \trans{p_\ast} \Sigma p_\ast \\
		&= \trans{(\lambda_1^\ast \mu + \lambda_2^\ast \one)} \Sigma^{-1} \Sigma \Sigma^{-1} (\lambda_1^\ast \mu + \lambda_2^\ast \one) \\
		&= (\lambda_1^\ast)^2 a + 2 \lambda_1^\ast \lambda_2^\ast b + (\lambda_2^\ast)^2 c \\
		&= \frac{1}{(ac - b^2)^2} \brackets{(c\mu_\ast - b)^2 a + 2(c \mu_\ast - b)(a - b \mu_\ast) b + (a- b \mu_\ast)^2 c} \\
		&= \frac{1}{ac - b^2} \brackets{c \mu _\ast^2 - 2b\mu_\ast + a^2}
	\end{aligned}
\end{equation*}

Der Graph von $(\sigma_\ast, \mu_\ast)$ ist ein Hyperbelast.
%TODO Graph

\pagebreak

\subsection{Markowitz-Modell II: optimale Anlage mit risikofreier Anlage}
Wir betrachten Anlagegüter $S = (S^1, \dots, S^n)$ mit stochastischen Einperioden-Renditen $R=\trans{(R^1, \dots, R^n)}$ und zusätzlich einer \textit{risikofreien} Anlage $S^0$ mit Verzinsung $r$.

Vermögen $w=1$ aufgeteilt auf $1 = p_0 + p_1 + \dots + p_n$. Wir setzen $p = \transpose{p_1, \dots, p_n} \in \Rn$. 

Erwartete Rendite
\begin{equation*}
	\mu_p = \EW[\trans{p} R + (1 - \trans{p}\one) r] = \trans{p} (\mu - r \one) + r
\end{equation*}
Risiko
\begin{equation*}
	\sigma_p = \sqrt{\Var[\trans{p}R]} = \sqrt{\trans{p} \Sigma p}
\end{equation*}
\fbox{\textbf{Anlageproblem}}
\begin{equation*}
	\begin{aligned}
		\min \, &\tfrac{1}{2} \trans{p} \Sigma p \qquad (p \in \Rn) \\
		\text{unter NB: } &\trans{p} (\mu - r\one) = \mu_\ast - r \qquad\text{(Zielrendite)}
	\end{aligned}
	\tag{Mark-2} \label{eq: mark-2}
\end{equation*}

Lösung: $\nearrow$ UE

optimales Portfolio
\begin{equation*}
	p_\ast = \underbrace{\lambda_\ast}_{\in \R} * \underbrace{\Sigma^{-1} (\mu - r \one)}_{\in \Rn} \mit \lambda_\ast = \frac{\mu_\ast - r}{a^2 - 2br + cr^2}
\end{equation*}

\begin{korollar}[Tobin's One-Fund-Theorem]
	Jedes Pareto-optimale Portfolio für \eqref{eq: mark-2} kann als Linearkombination der risikofreien Anlage und des Portfolios 
	\begin{equation*}
		\Sigma^{-1} (\mu - r \one)
	\end{equation*}
	dargestellt werden.
\end{korollar}

Graph von minimalem Risiko $\sigma_\ast$ und Zielrendite $\mu_\ast$:


\begin{*bemerkung}
	\begin{itemize}
		\item nominales Portfolio: $\theta = (\theta_1, \dots, \theta_n) \in \Rn$ ($\theta_i$ Stückzahl von Anlagegut $S^i$)
		\item Portfoliowert: $V_0 = \trans{\theta}$ und $S_0 = \sum_{i=1}^n \theta_i S_0^i \defqe w$ \dots Anfangskapital
		\item[]
		$V_T = \trans{\theta} S_T = \sum_{i=1}^n \theta_i S_T^i$
		
		\item relatives Portfolio: $p = (p_1, \dots, p_n) \in \Rn$ mit $p_i = \frac{\theta_i S_0^i}{w}$ \dots Vermögensanteil in $S^i$
		\begin{equation*}
			\sum_{i=1}^n p_i = \frac{1}{w} \sum_{i=1}^n \theta_i S_0^i = \frac{w}{w} = 1
		\end{equation*}
		
		\item Renditen --- Einzelnes Anlagegut: $R_i = \frac{S_T^i - S_0^i}{S_0^i}$ \\
		Gesamtes Portfolio: 
		\begin{equation*}
			\begin{aligned}
				R_p &= \frac{V_T - V_0}{V_0} = \frac{1}{w} \brackets{\sum_{i=1}^n \theta_i S_T^i - \theta_i S_0^i} \\
				&= \frac{1}{w} \sum_{i=1}^n \theta_i (S_T^i - S_0^i) \\
				&= \sum_{i=1}^n \underbrace{\frac{\theta_i S_0^i}{w}}_{= p_i} * R_i \\
				&= \sum_{i=1}^n p_i * R_i \\
				&= \trans{p} R
			\end{aligned}
		\end{equation*} 
		ist linear in $p$.
	\end{itemize}
\end{*bemerkung}